\documentclass{article}



\include{defs}
\usepackage{times}
% to avoid loading the natbib package, add option nonatbib:
% \usepackage[nonatbib]{neurips_2020} %% This allows for numbered references but citet will still work.
% \usepackage[numbers]{natbib}
% \usepackage[printonlyused,nohyperlinks]{acronym}
 \usepackage[printonlyused]{acronym}
\usepackage[export]{adjustbox} %adjustbox export is needed for one of the figures.
% \usepackage[export]{adjustbox}% http://ctan.org/pkg/adjustbox
\usepackage{algorithm}
\usepackage{algorithmicx}
\usepackage{algpseudocode}
\algrenewcommand\algorithmicindent{0.8em}%
\usepackage{amssymb}
\usepackage{amsmath}
\usepackage{array}
\usepackage{bbm}
\usepackage{booktabs}
\usepackage{caption}
\usepackage{float}
\usepackage{graphicx}
\usepackage[pagebackref=true]{hyperref}
\usepackage{lipsum}
\usepackage{microtype}
\usepackage{multicol}
\usepackage{multirow}
\usepackage{natbib}
\usepackage{nicefrac}
\usepackage{relsize}
\usepackage{siunitx}
\sisetup{round-mode = figures, round-precision = 3,scientific-notation = false}
\usepackage{subcaption}
\usepackage{wrapfig}
\usepackage{tikz}
\usetikzlibrary{shapes, arrows, positioning, fit, backgrounds}
\usetikzlibrary{calc}
\usetikzlibrary{decorations.pathmorphing}
\tikzstyle{block} = [rectangle, draw, fill=blue!20,
    text centered, rounded corners, minimum height=1em, node distance=1.5cm]
\tikzstyle{line} = [draw, -latex']
\tikzstyle{cloud} = [draw, rectangle,fill=red!20, node distance=1.5cm and 1cm,
    minimum height=1em]
\usepackage{verbatim}
\usepackage{xspace}
%%%%%%%%%%%%%%%


\usepackage{amsthm}
\algnewcommand{\IfThenElse}[3]{% \IfThenElse{<if>}{<then>}{<else>}
  \State \algorithmicif\ #1\ \algorithmicthen\ #2\ \algorithmicelse\ #3}
 \algnewcommand{\IfThen}[2]{% \IfThenElse{<if>}{<then>}{<else>}
  \State \algorithmicif\ #1\ \algorithmicthen\ #2}

% Attempt to make hyperref and algorithmic work together better:
\newcommand{\theHalgorithm}{\arabic{algorithm}}
%%%%%%%%



\usepackage[utf8]{inputenc} % allow utf-8 input
\usepackage[T1]{fontenc}    % use 8-bit T1 fonts
\usepackage{hyperref}       % hyperlinks
\usepackage{url}            % simple URL typesetting
\usepackage{booktabs}       % professional-quality tables
\usepackage{amsfonts}       % blackboard math symbols
\usepackage{nicefrac}       % compact symbols for 1/2, etc.
\usepackage{microtype}      % microtypography
%%%%%%%

\definecolor{mydarkblue}{rgb}{0,0.08,0.45}
\hypersetup{ %
    pdftitle={},
    pdfauthor={},
    pdfsubject={},
    pdfkeywords={},
    pdfborder=0 0 0,
    pdfpagemode=UseNone,
    colorlinks=true,
    linkcolor=mydarkblue,
    citecolor=mydarkblue,
    filecolor=mydarkblue,
    urlcolor=mydarkblue,
    pdfview=FitH}

\algnewcommand{\algorithmicforeach}{\textbf{for each}}
\algdef{SE}[FOR]{ForEach}{EndForEach}[1]
  {\algorithmicforeach\ #1\ \algorithmicdo}% \ForEach{#1}
  {\algorithmicend\ \algorithmicforeach}% \EndForEach
\newcommand{\pushright}[1]{\ifmeasuring@#1\else\omit\hfill$\displaystyle#1$\fi\ignorespaces}

\usetikzlibrary{tikzmark}
\usetikzlibrary{calc}
% begin vertical rule patch for algorithmicx (http://tex.stackexchange.com/questions/144840/vertical-loop-block-lines-in-algorithmicx-with-noend-option)
% note that some of the packages above are also needed
\newcommand{\ALGtikzmarkcolor}{black}% customise this, if you want
\newcommand{\ALGtikzmarkextraindent}{2pt}% customise this, if you want
\newcommand{\ALGtikzmarkverticaloffsetstart}{-.5ex}% customise this, if you want
\newcommand{\ALGtikzmarkverticaloffsetend}{-.5ex}% customise this, if you want
\makeatletter
\newcounter{ALG@tikzmark@tempcnta}

\newcommand\ALG@tikzmark@start{%
    \global\let\ALG@tikzmark@last\ALG@tikzmark@starttext%
    \expandafter\edef\csname ALG@tikzmark@\theALG@nested\endcsname{\theALG@tikzmark@tempcnta}%
    \tikzmark{ALG@tikzmark@start@\csname ALG@tikzmark@\theALG@nested\endcsname}%
    \addtocounter{ALG@tikzmark@tempcnta}{1}%
}

\def\ALG@tikzmark@starttext{start}
\newcommand\ALG@tikzmark@end{%
    \ifx\ALG@tikzmark@last\ALG@tikzmark@starttext
        % ignore this, the block was opened then closed directly without any other blocks in between (so just a \State basically)
        % don't draw a vertical line here
    \else
        \tikzmark{ALG@tikzmark@end@\csname ALG@tikzmark@\theALG@nested\endcsname}%
        \tikz[overlay,remember picture] \draw[\ALGtikzmarkcolor] let \p{S}=($(pic cs:ALG@tikzmark@start@\csname ALG@tikzmark@\theALG@nested\endcsname)+(\ALGtikzmarkextraindent,\ALGtikzmarkverticaloffsetstart)$), \p{E}=($(pic cs:ALG@tikzmark@end@\csname ALG@tikzmark@\theALG@nested\endcsname)+(\ALGtikzmarkextraindent,\ALGtikzmarkverticaloffsetend)$) in (\x{S},\y{S})--(\x{S},\y{E});%
    \fi
    \gdef\ALG@tikzmark@last{end}%
}



% the following line injects our new tikzmarking code
\apptocmd{\ALG@beginblock}{\ALG@tikzmark@start}{}{\errmessage{failed to patch}}
\pretocmd{\ALG@endblock}{\ALG@tikzmark@end}{}{\errmessage{failed to patch}}
%%%%% Annotation functions
\newcommand{\todo}[1]{\textcolor{red}{Todo:#1}}
% \newcommand{\todo}[1]{}
\newcommand{\glen}[1]{\textcolor{blue}{Glen:#1}}
\newcommand{\reviewercomment}[1]{\textcolor{orange}{ReviewComment:#1}}
\newcommand{\todocite}[1]{\textcolor{red}{[Cite:??#1??]}}
% \newcommand{\changes}[1]{{\color{blue}{#1}}}
\newcommand{\changes}[1]{#1}
\newcommand{\newchanges}[1]{\textcolor{red}{#1}}

% % % % % Words for refernce types % % % %\math{L}
\newcommand{\appendixx}[0]{supplementary document\xspace}
\newcommand{\function}[1]{\ensuremath{\textit{#1}}}
\newcommand{\procedure}[1]{\ensuremath{\textit{#1}}}
\newcommand{\valueWithUnits}[2]{#1~\normalsize #2\normalsize}
\newtheorem{theorem}{Theorem}[section]
% \newtheorem{theorem}{Theorem}[section]
\newtheorem*{theorem-nonlabeled}{Theorem}
\newtheorem{corollary}{Corollary}[theorem]
\newtheorem{lemma}[theorem]{Lemma}

%%%%% common math stuff %%%%%

\newcommand{\normalDistribution}[2]{\mathcal{N}(#1,#2)}
\newcommand{\expectation}{\mathbb{E}}
\newcommand{\lossFunction}[1]{\mathcal{L}(#1)}
\newcommand{\grad}[0]{\nabla}
\newcommand{\given}{\:\vert\:}
\newcommand{\placeholder}{\:\cdot\:}
\DeclareMathOperator*{\argmin}{arg\,min \:}
\DeclareMathOperator*{\argmax}{arg\,max \:}


% % % % % Words for things % % % % %


\newcommand{\videoLink}{https://youtu.be/s1KiIrV1YY4}
\newcommand{\methodName}{SMiRL\xspace}

\newcommand{\transitionTuple}{\tau}
% \newcommand{\nextState}{\sstate}
% \newcommand{\nextObservation}{\nextState}
\newcommand{\discountFactor}{\gamma}
\newcommand{\expNoise}{\lambda}
\newcommand{\experianceTuple}{\tau}
\newcommand{\experianceMemory}{D}
\newcommand{\learningRate}{\alpha}
\newcommand{\targetValue}{y}
\newcommand{\targetDifference}{\delta}




% use \acrodef to define an acronym, but no listing

% The acronym environment will typeset only those acronyms that were
% *actually used* in the course of the document
% \begin{acronym}[ANOVA]
\acrodef{AGI}{artificial general intelligence}
\acrodef{ANOVA}[ANOVA]{Analysis of Variance\acroextra{, a set of
  statistical techniques to identify sources of variability between groups}}
\acrodef{ANN}{artificial neural network}
\acrodef{API}{application programming interface}
\acrodef{CACLA}{continuous actor critic learning automaton}
\acrodef{cGAN}{conditional generative adversarial network}
\acrodef{CMA}{covariance matrix adaptation}
\acrodef{COM}{centre of mass}
\acrodef{CTAN}{\acroextra{The }Common \TeX\ Archive Network}
\acrodef{DDPG}{deep deterministic policy gradient}
\acrodef{DeepLoco}{deep locomotion}
\acrodef{DOI}{Document Object Identifier\acroextra{ (see
    \url{http://doi.org})}}
\acrodef{DPG}{deterministic policy gradient}
\acrodef{DQN}{deep Q-network}
\acrodef{DRL}{deep reinforcement learning}
\acrodef{DYNA}{DYNA}
\acrodef{EOM}{Equations of motion}
\acrodef{EPG}{expected policy gradient}
\acrodef{FDR}{future discounted reward}
\acrodef{FSM}{finite state machine}
\acrodef{GAE}{generalized advantage estimation}
\acrodef{GAIfO}{generative adversarial imitation from observation}
\acrodef{GAN}{generative adversarial network}
\acrodef{GPS}[GPS]{Graduate and Postdoctoral Studies}
\acrodef{HLC}{high-level controller}
\acrodef{HLP}{high-level policy}
\acrodef{HRL}{hierarchical reinforcement learning}
\acrodef{KLD}{Kullback-Leibler divergence}
\acrodef{LLC}{low-level controller}
\acrodef{LLP}{low-level policy}
\acrodef{MARL}{Multi-Agent Reinforcement Learning}
\acrodef{MBAE}{model-based action exploration}
\acrodef{MPC}{model predictive control}
\acrodef{MDP}{Markov Decision Processes}
\acrodef{MSE}{mean squared error}
\acrodef{MultiTasker}{controller that learns multiple tasks at the same time}
\acrodef{Parallel}{randomly initialize controllers and train in parallel}
\acrodef{PD}{proportional derivative}
\acrodef{PDF}{Portable Document Format}
\acrodef{PLAiD}{Progressive Learning and Integration via Distillation}
\acrodef{PPO}{proximal policy optimization}
\acrodef{PTD}{positive temporal difference}
\acrodef{RBF}{radial basis function}
\acrodef{ReLU}{rectified linear unit}
\acrodef{RCS}[RCS]{Revision control system\acroextra{, a software
    tool for tracking changes to a set of files}}
\acrodef{RL}{reinforcement learning}
\acrodef{SGD}{stochastic gradient descent}
\acrodef{Scratch}{randomly initialized controller}
\acrodef{SIMBICON}{SIMple BIped CONtroller}
\acrodef{SMBAE}{stochastic model-based action exploration}
\acrodef{SVG}{stochastic value gradients}
\acrodef{SVM}{support vector machine}
\acrodef{TCN}{time contrastive learning}
\acrodef{TD}{temporal difference}
\acrodef{TL}{transfer learning}
\acrodef{terrainRL}{terrain adaptive locomotion}
\acrodef{TLX}[TLX]{Task Load Index\acroextra{, an instrument for gauging
  the subjective mental workload experienced by a human in performing
  a task}}
\acrodef{TRPO}{trust region policy optimization}
\acrodef{UBC}{University of British Columbia}
\acrodef{UCB}{upper confidence bound}
\acrodef{UI}{user interface}
\acrodef{UML}{Unified Modelling Language\acroextra{, a visual language
    for modelling the structure of software artefacts}}
\acrodef{URDF}{unified robot description format}
\acrodef{URL}{Unique Resource Locator\acroextra{, used to describe a
    means for obtaining some resource on the world wide web}}
\acrodef{W3C}[W3C]{\acroextra{the }World Wide Web Consortium\acroextra{,
    the standards body for web technologies}}    
\acrodef{XML}{Extensible Markup Language}
% \end{acronym}

	% always input, since other macros may rely on it
\title{Title}


\author{ 
  }
\begin{document}
% \nipsfinalcopy is no longer used
%\input{icml2020_header.tex}
\maketitle

% use \acrodef to define an acronym, but no listing

% The acronym environment will typeset only those acronyms that were
% *actually used* in the course of the document
% \begin{acronym}[ANOVA]
\acrodef{AGI}{artificial general intelligence}
\acrodef{ANOVA}[ANOVA]{Analysis of Variance\acroextra{, a set of
  statistical techniques to identify sources of variability between groups}}
\acrodef{ANN}{artificial neural network}
\acrodef{API}{application programming interface}
\acrodef{CACLA}{continuous actor critic learning automaton}
\acrodef{cGAN}{conditional generative adversarial network}
\acrodef{CMA}{covariance matrix adaptation}
\acrodef{COM}{centre of mass}
\acrodef{CTAN}{\acroextra{The }Common \TeX\ Archive Network}
\acrodef{DDPG}{deep deterministic policy gradient}
\acrodef{DeepLoco}{deep locomotion}
\acrodef{DOI}{Document Object Identifier\acroextra{ (see
    \url{http://doi.org})}}
\acrodef{DPG}{deterministic policy gradient}
\acrodef{DQN}{deep Q-network}
\acrodef{DRL}{deep reinforcement learning}
\acrodef{DYNA}{DYNA}
\acrodef{EOM}{Equations of motion}
\acrodef{EPG}{expected policy gradient}
\acrodef{FDR}{future discounted reward}
\acrodef{FSM}{finite state machine}
\acrodef{GAE}{generalized advantage estimation}
\acrodef{GAIfO}{generative adversarial imitation from observation}
\acrodef{GAN}{generative adversarial network}
\acrodef{GPS}[GPS]{Graduate and Postdoctoral Studies}
\acrodef{HLC}{high-level controller}
\acrodef{HLP}{high-level policy}
\acrodef{HRL}{hierarchical reinforcement learning}
\acrodef{KLD}{Kullback-Leibler divergence}
\acrodef{LLC}{low-level controller}
\acrodef{LLP}{low-level policy}
\acrodef{MARL}{Multi-Agent Reinforcement Learning}
\acrodef{MBAE}{model-based action exploration}
\acrodef{MPC}{model predictive control}
\acrodef{MDP}{Markov Decision Processes}
\acrodef{MSE}{mean squared error}
\acrodef{MultiTasker}{controller that learns multiple tasks at the same time}
\acrodef{Parallel}{randomly initialize controllers and train in parallel}
\acrodef{PD}{proportional derivative}
\acrodef{PDF}{Portable Document Format}
\acrodef{PLAiD}{Progressive Learning and Integration via Distillation}
\acrodef{PPO}{proximal policy optimization}
\acrodef{PTD}{positive temporal difference}
\acrodef{RBF}{radial basis function}
\acrodef{ReLU}{rectified linear unit}
\acrodef{RCS}[RCS]{Revision control system\acroextra{, a software
    tool for tracking changes to a set of files}}
\acrodef{RL}{reinforcement learning}
\acrodef{SGD}{stochastic gradient descent}
\acrodef{Scratch}{randomly initialized controller}
\acrodef{SIMBICON}{SIMple BIped CONtroller}
\acrodef{SMBAE}{stochastic model-based action exploration}
\acrodef{SVG}{stochastic value gradients}
\acrodef{SVM}{support vector machine}
\acrodef{TCN}{time contrastive learning}
\acrodef{TD}{temporal difference}
\acrodef{TL}{transfer learning}
\acrodef{terrainRL}{terrain adaptive locomotion}
\acrodef{TLX}[TLX]{Task Load Index\acroextra{, an instrument for gauging
  the subjective mental workload experienced by a human in performing
  a task}}
\acrodef{TRPO}{trust region policy optimization}
\acrodef{UBC}{University of British Columbia}
\acrodef{UCB}{upper confidence bound}
\acrodef{UI}{user interface}
\acrodef{UML}{Unified Modelling Language\acroextra{, a visual language
    for modelling the structure of software artefacts}}
\acrodef{URDF}{unified robot description format}
\acrodef{URL}{Unique Resource Locator\acroextra{, used to describe a
    means for obtaining some resource on the world wide web}}
\acrodef{W3C}[W3C]{\acroextra{the }World Wide Web Consortium\acroextra{,
    the standards body for web technologies}}    
\acrodef{XML}{Extensible Markup Language}
% \end{acronym}

	% always input, since other macros may rely on it

%%%%%%%%%%%%%%%%%%%%%%%%%%%%%%%%%%%%%%%%%%%%%%%%%%%%%%%%%%%%%%%%%%%%%%%%%%%
\setlength\abovecaptionskip{0.1cm}



\begin{abstract}
paper summary
\end{abstract}

\section{Introduction}
\label{sec:intro}

We like \ac{MDP}s

%%GB.4.24.23: This introduction should work on following the general guidelines for writing an introduction that lists the paragraph in order:
% First, what is the problem, and why is it important? 
% Second, why is that problem hard? Why hasn't it already been solved?
% Third, What is a proposed methodological solution (high-level) to the method, and why is it a good proposed method?
% Fourth, describe your proposed technical solution and how it implements the method described in paragraph three.
% five, discuss your technical contributions.

\section{Related Work}
\label{sec:intro}

Example citation
\citep{mpo}

\section{Background}
\label{sec:background}

In this section, we provide a very brief review of the fundamental background used by our method. \ac{RL} is formulated within the framework of a \ac{MDP} where at every time step $t$, the world (including the agent) exists in a state $ \bs_{t} 
\in \states $, where the agent is able to perform actions $ \ba_{t} \in 
\actions $. The action to take is determined according to a policy $ \pi(\ba_{a}|\bs_{t})$ which
results in a new state $ \bs_{t+1} \in \states $  and reward $\reward_{t} = R(\bs_t, \ba_t)$ according to the transition 
probability function $ P(\bs_{t+1} | \bs_{t}, \bs_{t}) $. 
The policy is optimized to maximize the future discounted reward
%
% \begin{equation}
% \label{eq:policy-gradient}
% J(\policySymbol) =
$
  \expectation_{\reward_{0}, ..., \reward_T} \left[ \sum_{t=0}^{T} \gamma^t \reward_{t} \right]$,
% \end{equation}
\noindent where $ T $ is the max time horizon, and $ \discountFactor $ is the 
discount factor.
The formulation above generalizes to continuous states and actions, which is the situation for the agents we consider in our work.

\section{Method Name}
\label{sec:method} 

Some text

\begin{wrapfigure}{R}{0.5\textwidth}
\vspace{-0.25cm}
\begin{minipage}{.5\textwidth}
\begin{algorithm}[H]
\footnotesize 	
\caption{\methodName}
\label{alg:training}
\begin{algorithmic}[1]
\While{not converged}
\State $\beta\gets \{\}$ \algorithmiccomment{Reset experience}
\For{$\text{episode } = 0,\dots,M$}
\State $\bs_0 \sim p(\bs_0); \tau_0 \gets  \{\bs_0\}$ \algorithmiccomment{Initialize state}
\State $\bar{\bs}_0 \gets (\bs_0, \mathbf{0}, 0)$ \algorithmiccomment{Initialize aug. state}
\ForEach{$t = 0,\dots,T$}
\State $a_{t} \sim \pi_\phi(a_t| \bs_{t}, \theta_{t}, t)$\algorithmiccomment{Get  action}
\State $\bs_{t+1} \sim T(\bs_{t+1} | \bs_{t}, a_{t})$  \algorithmiccomment{Step dynamics}
\State $r_{t} \gets  \log p_{\theta_{t}}(\bs_{t+1})$ \algorithmiccomment{\methodName reward}
\State $\tau_{t+1}\!\gets\!\tau_t\cup \{\bs_{t+1}\}$ \algorithmiccomment{Record state}
\State $\theta_{t+1} \gets \upd(\tau_{t+1})$ \algorithmiccomment{Fit model}
\State $\bar{\bs}_{t+1} \gets \{( \bs_{t+1}, \theta_{t+1}, t_{t+1})\}$
\State $\beta \gets \beta \cup \{(\bar{\bs}_{t}, a_t, r_t, \bar{\bs}_{t+1})\}$
\EndFor
\EndForEach
\State $\phi \gets \texttt{RL} (\phi, \beta)$ \algorithmiccomment{Update policy}
\EndWhile
\end{algorithmic}
\end{algorithm}
% \end{minipage}
\end{minipage}
\vspace{-0.25cm}
\end{wrapfigure}

\section{Evaluation Environments}
\label{sec:envs}

\section{Experimental Results}
\label{sec:results}

\begin{figure*}[t!]
% \vspace{0.5cm}
\centering
%trim=l b r t
\subcaptionbox{\label{fig:humanoid2d-compare} humanoid2d walk}{ \includegraphics[trim={0.0cm 0.0cm 0.0cm 0.0cm},clip,width=0.31\linewidth]{images/todo.png}}
\subcaptionbox{\label{fig:dog2d-compare}  dog2d}{ \includegraphics[trim={0.0cm 0.0cm 0.0cm 0.0cm},clip,width=0.31\linewidth]{images/todo.png}} 
\subcaptionbox{\label{fig:raptor2d}  raptor2d}{ \includegraphics[trim={0.0cm 0.0cm 0.0cm 0.0cm},clip,width=0.31\linewidth]{images/todo.png}}
\caption{
Comparisons of \methodName, \ac{TCN} and \ac{GAIfO} with (a) the humanoid2d, (b) the 2D dog agent, and (c) the 2D raptor agent. \ac{GAIfO} struggles to show improvement on these tasks. \ac{TCN} does make progress on these imitation tasks but the performance is not as good as \methodName.
The results show average performance over $4$ randomly seeded policy training simulations.
%The dotted lines of the same colour are the specific performance values for each policy training run.
}
\label{fig:humanoid2d-rl-compare-old-cd}
\vspace{-0.25cm}
\end{figure*}


\section{Discussion}
\label{sec:discussion}


\bibliographystyle{plainnat}
\bibliography{references} 
\clearpage
\appendix

\end{document}
